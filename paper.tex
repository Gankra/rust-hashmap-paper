\documentclass{cccg13}

\usepackage{graphicx,amssymb,amsmath,dsfont}

% URLs
\usepackage{url}

%comments
\usepackage{verbatim}

\newcommand{\IR}{\mathbb{R}}
\newcommand{\C}{\mathcal{C}}
\newcommand{\MP}{\mathcal{P}}
\newcommand{\T}{\mathcal{T}}
\newcommand{\F}{\mathcal{F}}
\newcommand{\CH}{Conv}

\title{Implementing A Safe, Efficient, and Secure Hash table in Rust \thanks{A.B. was supported by NSERC.}}

\author{
Alexis Beingessner\footnotemark[2]
\and
Piotr Czarnecki
\and
Niko Matsakis
}

\begin{document}
\thispagestyle{empty}
\maketitle

%%%%%%%%%%%%%%%%%%%%%%%%%%%%%%%%%%%
%%%%%%%%%%%%ABSTRACT%%%%%%%%%%%%%%%
%%%%%%%%%%%%%%%%%%%%%%%%%%%%%%%%%%%

\begin{abstract}
In this paper we describe the strategies used in Rust's standard collections library to implement a robin hood hash table that is memory-safe, thread-safe, fast, and robust against denial of service attacks.
\end{abstract}






%%%%%%%%%%%%%%%%%%%%%%%%%%%%%%%%%%%%%
%%%%%%%%%%%INTRO%%%%%%%%%%%%%%%%%%%%%
%%%%%%%%%%%%%%%%%%%%%%%%%%%%%%%%%%%%%

\section{Introduction}   \label{secintro}

Hash tables are one of the most important and common data structures used by
developers across almost every programming language. They provide excellent
practical and theoretical performance, while being relatively simple to write
and use. However, because of this ubiquity, any implementation provided by a
programming language must be \emph{exceptionally} well written. First and
foremost, the implementation must be exceptionally \emph{efficient}, or else
it simply won't be used. However, since hash tables are regularly used to
process user input, a standard implementation must also be \emph{secure}
against certain kinds of \emph{denial of service attacks} (also known as
\emph{algorithmic complexity attacks} and \emph{hash flooding attacks}).

In a denial of service attack, a malicious end-user provides data to a service
which will somehow trigger degenerate behaviour in a hash table. The simplest
attack is to produce a series of values that all hash to the same number. In
most hash table implementations, this will result in disastrous performance:
search will be reduced to linearly walking over every item in the table,
causing operations that are intended to take $O(1)$ time to take $O(n)$ time
instead. Inserting all of these elements would consequently take
$O(n^2)$. This can cause an otherwise well-written application to become
unresponsive, denying service to real users.

These kind of attacks were first described in detail in 2003 by Crosby and
Wallach \cite{perl-dos} when they demonstrated attacks against the Perl
programming language and several applications. Klink and Wälde \cite{web-dos}
popularized this kind of attack in 2011 when they published an attack against
the hash table implementations in PHP, Java, Ruby, and several other languages
that were able to lock up some popular web frameworks using relatively small
POST requests.

The solution to this specific attack is simple and well-known: use a
\emph{randomized} hash function.  Generally, this is done by using a single
hashing algorithm that incorporates a secret key in the hashing process. A
random hash function is chosen by generating a random key per process, thread,
or hash table instance. With a sufficiently large key, this should in theory
prevent an attacker from identifying a sequence of values that hash to the
same number, as a collision in one function is unlikely to be a collision in
another random function. However some care must be taken, as a sufficiently
motivated attacker may be able to identify which function was chosen by
probing the application's behaviour. In 2012, attacks against MurmurHash and
CityHash were presented by Aumasson, Bernstein, and Boßlet \cite{murmur-dos}.
This is in spite of the fact that these hashing algorithms \emph{were}
randomized.

Even if a good hash function is chosen, collisions are still expected to occur
in hash tables, and they must be handled. There are two primary families of
collision resolution strategies: chaining, and open addressing. In chaining
hash tables, each index in the table contains a pointer to an \emph{auxiliary}
structure for storing all values with that hash. The simplest auxiliary
structure is a singly-linked list, although any structure can be used. With a
robust hash function, chaining tables are fully secure. The only way to
produce degenerate behaviour is to produce a large collection of values that
all have the exact same hash, which is exactly what the hash function should
protect against. In fact, a chaining hash table can be secured against denial
of service \emph{without} a secure hash function by making the auxiliary
structure a binary search tree.

Unfortunately, chaining is not an appropriate choice for a standard hash table
because they generally fail to satisfy the \emph{first} requirement:
efficiency. The auxiliary structure adds significant indirection and overhead
to the search process. Extra space is consumed just to represent the structure
of the table, where ideally the table can just be a single contiguous block of
memory. Node-based auxiliary structures in particular also suffer from poor
cache-efficiency, and require an allocation for every insertion, which is
notoriously slow to do. Using a binary search tree also adds the additional
requirement that the keys to be stored in the table are \emph{orderable}.

Open addressing tables avoid these problems by instead trying to find a
location in the table that isn't yet occupied. By doing this, the table
\emph{can} be represented by a single block of memory, allocations only need
to be performed to resize the table, and cache efficiency can potentially be
improved.

The simplest open addressing scheme is \emph{first-come-first-serve} (FCFS)
linear search. In this scheme, if an index is occupied the search procedure
simply walks to the right until it finds an empty index (indexing is assumed
to be circular in this case). A similar procedure is \emph{last-come-first-
serve} (LCFS) linear search, in which the element to be inserted kicks any
pre-existing element out of the spot, and tells it to try to insert itself in
the next index in the table.

Practical performance for these schemes can be excellent. Modern CPUs excel at
linear traversals of memory. Further, the average distance traveled for each
insertion can be quite low for these schemes. However, they have a critical
security issue: search times have a very high variance. If an attacker is able
to trigger queries for an element on-demand, then they need only produce a
\emph{single} element that takes a very long time to search for to perform a
denial of service attack against a hash table.

\emph{Robin hood hashing} (RHH) is another open addressing scheme similar to
FCFS and LCFS that seeks to produce ``fair" search times for all elements.
This potentially increases the search times for some elements in order to
reduce the search times for others. Like the other schemes, RHH searches
linearly for an appropriate location. However RHH also tracks how far an
element is from the \emph{ideal bucket} (IB) that its hash specifies. In the
case of a collision, priority is given to the element that currently has the
\emph{greatest} distance from its ideal bucket (DIB). In this way, robin hood
hashing ``takes from the rich, and gives to the poor". Using this scheme, much
tighter clustering is required to produce degenerate behaviour than is
necessary in FCFS or LCFS.

[TODO diagram: FCFS, LCFS, and RHH on the insertion hash sequence 1,1,2,3,2,1.
In FCFS the last inserted element has every element in the table between it
and its ideal location; In LCFS the first inserted element has every element
in the table between it and its ideal location; In RHH all the distances are
more even.]

In FCFS and LCFS, it is also possible to trigger degenerate behaviour without
any element being particularly far from its ideal location. Instead, if one
can construct a large contiguous block of occupied buckets, one can force the
entire block to be traversed by searching for an element \emph{not} found in
the table. In FCFS and LCFS, the only way to be certain that an element isn't
in the table is to run into an empty bucket. RHH by contrast provides a
mechanism to terminate searches for an element that isn't in the table before
an empty space is found: If at any point during the search for an element $x$,
if an element $y$ is encountered with lower DIB than the current search
distance, search can be immediately terminated because if $x$ \emph{were} in
the table, the robin hood scheme would have had $x$ displace $y$. Therefore,
the only way to create a search of length $k$ is to have an actual element
with DIB $k$.

Deletion in RHH can effectively reverse the process of insertion. Once the
element to delete is found and removed, the elements after it are back-shifted
by one bucket until an element in its ideal bucket or an empty space is
encountered. It is as if the element was never inserted, and therefore all the
properties of insertion and search are trivially preserved.





\section{Rust}

Rust is an experimental systems programming language under development by
Mozilla. Most notable is its ability to provide much stronger memory safety
guarantees than most other procedural programming languages. In particular, it
has a simple effect system that separates code into two classes: safe code,
and unsafe code.

In safe code, if there exists a reference, whether or not mutable, to a
location in memory, there can be no other way of \emph{mutating} that memory.
Further, all references must
always be valid; null or dangling pointers cannot occur. These properties,
among others, effectively eliminate most memory-access and data-race errors.
In addition, any structure that obeys these rules is thread-safe ``for free",
because being thread-\emph{unsafe} requires a mutable reference and another
reference to co-exist, which is forbidden by safe code. However, these
constraints often aren't trivial to work with. Particularly when implementing
data structures, which are often designed with C-like total access in mind.

Unsafe code allows the programmer to bypass the mechanisms that enforce these
constraints locally. This is done using an ``unsafe" keyword to delimit
where the invariants of safe code are guaranteed to be upheld. In this way,
users of Rust can primarily work in safe code, where segmentation faults are
impossible and uninitialized memory is inaccessible, but can temporarily drop
down to unsafe code to achieve the level of control found in C or C++ where
deemed necessary.

A Rust programmer's goal is then to minimize the lines of code that are
declared unsafe, so as to minimize the places where errors can occur. Further,
unsafe code should be primarily relegated to internal implementation details.
External APIs should be completely safe. Although unsafe APIs may also be
exposed for lower-level access, it should never be *necessary* for a user to
write unsafe code to perform a normal operation. For instance, array indexing
in Rust is bounds-checked by default, but there exists an unsafe unchecked
variant for when performance is critical, and the programmer is certain that
they know what they're doing. Ideally, these methods would only be used
\emph{after} performance issues are identified.

Due to the performance-critical nature of data structures, the use of unsafe
code is inevitable. Raw uninitialized chunks of memory are a standard building
block for array-based structures, and even the back-pointers on a doubly-
linked list are considered unsafe by Rust. The goal of data structure design
in Rust has thus become to relegate unsafe code to a minimal substructure with
a safe interface, to be used by the primary structure.

Interface design has recently gravitated towards providing several wrapper
objects which effectively put the structure into different access ``modes". By
constraining the set of valid behaviours, operations which would otherwise be
unsafe to provide become safe. Rust's \emph{iterator} interface is the easiest
understood example of this.

Usually in safe code, acquiring a mutable reference into the contents of a
structure effectively locks the structure. While such a reference exists, no
other internal references can be acquired because the user could otherwise
request the same item multiple times, invalidating Rust's multiple-aliasing
restriction. The structure also cannot be modified, because this could
invalidate the reference and produce a dangling pointer. However, a
collection's \emph{iterator} wraps the object and exposes only one operation
called \emph{next} that yields a reference to some internal element that has
not yet been yielded. Since the iterator guarantees elements are only yielded
once, and no other operations are permitted at this time, it is safe to obtain
multiple internal mutable references. Certain mutations can also be safely
performed while holding references from an iterator. For instance, it is
perfectly safe to delete the element that \emph{would} next be yielded by a
linked list's iterator, because this won't effect all the previously yielded
references.




\section{Rust's Robin Hood Hash Table}

Rust's standard library of course has an implementation of a hash table. After
several iterations of design and implementation, we have converged on a robin
hood hash table design as described in the introduction. While this paper
describes the implementation in detail, we will not be providing the full code
here for the sake of brevity. Such code would also become hopelessly out of
date in a short amount of time, as the language is still under active
development. If the reader is interested in reviewing the actual code, they
can freely view the code on Github. \cite{rust-source}

Our hash table implementation is separated into three modules: map, set, and
table. Table is the \emph{unsafe} portion of the implementation. It handles
the allocation and manipulation of the raw block of memory, which is exposed
through a RawTable type. RawTable's simplicity makes it technically
appropriate for implementing different kinds of open addressing hash tables,
although this has not been done. Map provides all of the actual \emph{logic}
of the implementation. It provides the interface for our HashMap type, and
performs all of the robin hood scheme's logic on top of a RawTable. Finally,
set provides a HashSet type which is a thin convenience wrapper around
HashMap. Thanks to Rust's type system, this is substantially more efficient
than possible in other languages.





\section{Hashing}
For several years, Rust's hash table implementation had used SipHash 2-4
\cite{siphash} as the default hashing algorithm. SipHash was specifically
designed with preventing hash table denial of service attacks in mind. It has
excellent theoretical and experimental properties for preventing an attacker
from reliably producing large hash collisions. However, SipHash has a few
performance issues. It requires 128 bits of entropy for its private key, which
can be expensive if each table has its own SipHash instance. It also requires
a certain minimum amount of work to be done in its finalization step, making
small keys relatively expensive to hash.

SipHash was deemed too slow for the kind of workloads found in the Rust
compiler itself. Thankfully, our hash table is generic over arbitrary
streaming hash functions. For these workloads, SipHash was replaced with
FNV, which provides excellent performance and nice distributions on random
inputs, but no cryptographic guarantees. This was deemed an acceptable risk
for the sake of productivity, as denial of service attacks against a compiler
were considered low risk. Or at least, compilation speed was considered a
higher priority. However, this produced an awkward schism between secure and
performance-critical applications. All applications should ideally be secure,
and we certainly don't want people throwing away security in the name of
performance.

To solve this, we adopted a hybrid hashing strategy. Today, Rust's hash tables
default to using xxHash, but with a special fallback mode to SipHash. If the
search for any key takes over 64 steps while using xxHash, we check the
current load factor of the table. If the table is more than 63\% full, then we
assume that we have had some bad luck, and grow the table. If the table isn't
that full, then we assume we are being attacked, and rebuild the table from
scratch using SipHash. This of course incurs a substantial one-time cost, but
it's a linear cost, and not the quadratic cost that a denial of service attack
is after. There is also some overhead over \emph{just} using xxHash in some
synthetic benchmarks, but it's a relatively small cost. There are probably
richer areas for optimization elsewhere.





\section{Resizing}

Consider growing a hash table of capacity $n$. Normally, one would do this by
allocating a new table of capacity $2n$, and individually reinserting each
element in the old table into the new one using full insertion scheme. This
guarantees that the new table is valid and all essential invariants hold.
Remark that the order we reinsert the elements in should not
matter. For simplicity and efficiency, we will consider only \emph{linear}
reinsertions, which consist of reinserting all elements in the old table into
the new one by increasing order of index.

Recall that our hash generation scheme consists of generating a 64-bit hash
and truncating the most significant bits to acquire the element's ideal
bucket. When moving to the new table, we simply start using the next most
significant bit in the hash. Therefore, if an elements has ideal index $i$ in
the old table, it can have one of two ideal locations in the new table. If the
new bit is $0$, then the new ideal index is $i$. If the new bit is $1$, then
the new ideal index is $n + i$. Intuitively, an element's ideal index doesn't
change at all; it is just reinserted into one of two independent tables of
size $n$. Of course, this isn't quite true due to our circular indexing
scheme, however we will see that this isn't an issue.

\begin{lemma} \label{lem:exists-ideal}
  Let an \emph{ideal} element be an element in its ideal bucket. Then there
  always exists an ideal element in any non-empty robin hood hash table.
\end{lemma}

\begin{proof}
  If there is an empty bucket in the map, then the element after it must be
  ideal. If there is not an empty bucket in the map, then there \emph{was} one
  right before the last element was inserted. Remark that there is no ideal
  location in the map that the last element can choose in order to displace
  the ideal element after the previously empty bucket. It cannot directly
  displace the ideal element, as older elements win in a tie, and any shifting
  will stop as soon as an element ends up in the empty bucket. Therefore,
  there will always be an ideal element.
\end{proof}

\begin{defn}
  Let a \emph{chain} be a maximal sequence of contiguous elements starting at
  an ideal element, and terminating before an empty bucket or another ideal
  element. All elements are in exactly one chain, which we identify by its
  ideal element.
\end{defn}

\begin{defn}
  Let the \emph{first ideal element} be the ideal element in the table with
  lowest index. By Lemma \ref{lem:exists-ideal}, the first ideal element
  always exists.
\end{defn}

We would like to avoid having to perform the full robin hood reinsertion
scheme. We would like to instead use the first-come-first-serve reinsertion
scheme, as this is fairly minimal with respect to reads and writes performed.
We will accomplish this by starting FCFS reinsertion at the first ideal
element. We do this to ensure that we start reinsertion at the start of a
chain, which is necessary for correctness. As we will show, this produces
identical results to performing the full robin hood scheme.

Without loss of generality, we will assume the first ideal element is at index
$0$ in the table. This is fine to do because our indexing is circular, and
therefore we can arbitrarily redefine the index of this element to be index
$0$ without actually changing behaviour. Since the first ideal element has
index $0$, no elements have a circularly wrapped index.

\begin{lemma} \label{lem:shrinking-index}
  If an element was located at index $i$ in the original table, then in
  whichever of the two new tables an element is inserted, its final index must
  be strictly $\leq i$
\end{lemma}

\begin{proof}
  The only difference between the element's new table and the old table is
  that some of the elements may have been moved to the other table. Therefore,
  there is no way for an element to be pushed further from its ideal bucket
  than it already was in the old table. Further, an element's new index will
  certainly never drop below its ideal index, which can never be less than
  $0$. Therefore its index also cannot increase by decreasing enough to
  circularly wrap.
\end{proof}

\begin{lemma} \label{lem:still-ideal}
  After a linear robin hood reinsertion, all previously ideal elements are
  \emph{still} ideal.
\end{lemma}

\begin{proof}
   As an ideal element, the only way to become non-ideal in a robin hood hash
   table is to have an element with a smaller ideal index attempt to insert
   itself in your ideal bucket. Every previously ideal element did not have
   this occur in the old table, and by lemma \ref{lem:shrinking-index} we know
   that elements only ``go backwards" in the table. Therefore this will never
   happen.
\end{proof}

\begin{theorem}
  A linear robin hood reinsertion from the first ideal element produces
  identical results to a linear FCFS reinsertion from the same element.
\end{theorem}

\begin{proof}
  We will prove this by induction on the number of elements inserted into the
  new table.

  Base case ($k = 1$): There can be no collisions if only one element has been
  inserted, so this follows trivially.

  Inductive Step: Assume that we have inserted $k$ elements into the new
  table, and wish to insert a new element $x$.

  If $x$ was ideal in the old table, then by lemma \ref{lem:still-ideal} it
  will never be successfully challenged during robin hood reinsertion. By our
  inductive hypothesis, all the elements currently in the new table are
  exactly as they would be if they were inserted using robin hood reinsertion.
  Therefore, the ideal bucket for $x$ must be empty, and it will be inserted
  in its ideal location as in robin hood reinsertion.

  If $x$ was \emph{not} ideal in the old table, let $i$ be the ideal index of
  $x$ in the old table, and $d > 0$ be its old distance from $i$. Then in the
  old table $x$ was part of a chain of in which there were $d$ elements with
  ideal index $\leq i$ before it, as this is the only way to displace an
  element $d$ indices from its ideal bucket. All $d$ of these elements were
  already inserted into the new table, because we are inserting elements in
  increasing order. By our inductive hypothesis they are all in exactly the
  same location they would have been in by robin hood reinsertion. Some of
  them may be part of the ``other" half of the new table, but some of them may
  be in this half. All of the elements that were put in this half of the table
  still have an ideal location less than $x$'s, and therefore would all
  ``beat" $x$ in a robin hood challenge. Therefore $x$ will be inserted in the
  same place using FCFS reinsertion as robin hood reinsertion.
\end{proof}

\section{conclusion}

TODO

\addcontentsline{toc}{chapter}{References}
\small
\bibliographystyle{abbrv}
\bibliography{bibliography}

\end{document}


